\section{Baby Derangements: A Proof}

The last time when we left off on our baby derangement journey supported by
Lester, we did a little bit of fun combinatorics to figure out a closed form
for the number baby derangements with a certain number \( k \) matches for a
generalized \( n \) number of babies. Near the end, we wanted to show that the
definition was well defined (i.e. the sum of all baby derangements combines to
give us all \( n! \) arrangements) and that the expected value of matches
equals \( 1 \) independent of \( n \). I wasn't able to give a proof of it that
day because I didn't stare at the screen long enough, and so I defaulted to
using a bit of handwaving and saying that it held empirically.

During a bit of a math scrolling spree at 10:00 PM a few days ago, I found that
the expected value proof is actually in fact an IMO problem, specifically
problem 1 from 1987. \MarginComment{Sure it's only problem 1 on the IMO, but
it's an IMO problem nonetheless, so it's still fun.} This was the perfect
motivation to actually go ahead and work on the proof. So without further
ado, here's a hopefully self-contained and rigorous \textit{enough}
solution.

\begin{blackbox}
    \begin{problem}[IMO 1987 \#1]
        Let \( p_n (k) \) be the number of permutations of the set \( \{1,
        \ldots, n\} \), \( n \geqslant 1 \), which have exactly \( k \) fixed
        points. Prove that
        \[
            \sum_{k = 0}^{n} k p_n (k) = n!
        .\]
    \end{problem}
\end{blackbox}

Although it may not look like it at first glance, this is exactly equivalent to
our expected value baby derangements problem. Instead of working with babies
matching to their mothers, we work instead with the more "dry" fixed points of
permutations, but it is the same underlying concept. One can more intuitively
transform the problem to the expected value one by dividing both sides by the
\( n! \).

It's time to fully tackle the problem now. We'll need a small tool to start.
\begin{proposition}
    The number of permutations of the set \( \{ 1, \ldots, n \} \) with exactly \( k \) fixed points, \( p_n (k) \), is given by
    \[
        p_n(k) = \binom{n}{k} D_{n - k}
    ,\]
    where \( D_m \) denotes the number of derangements of \( m \) items.
\end{proposition}
\begin{proof}
    We shall prove this by a straightforward combinatorial argument.

    Fix \( k \) items out of the set. We must necessarily permute the remaining
    \( n - k \) items such that they have no fixed point, precisely the
    definition of a derangement. Since there are \( \binom{n}{k} \) ways to fix
    \( k \) elements and \( D_{n - k} \) ways to permute \( n - k \) elements
    to have no fixed point, we have that
    \[
        p_n (k) = \binom{n}{k} D_{n - k}
    .\]
\end{proof}
With this, we can move on to the actual proof of the problem, which is really
just a decent bit of sum manipulation.
\begin{proof}
    As previously mentioned, the problem statement is equivalent to proving that
    \[
        \sum_{k = 0}^{n} k \frac{p_n (k)}{n!} = 1
    .\]
    For this, we shall use a form of induction. Let \( f(n) \) denote the LHS
    of the previous equation. We can see quite trivially that \( f(1) = 0 + 1 =
    1 \), so all that's left is to prove \( f(n + 1) - f(n) = 0 \) for \( n \geqslant 1 \). In other words, we desire
    \begin{align*}
        \left( \sum_{k = 0}^{n + 1} k \frac{p_{n + 1}(k)}{(n+1)!} \right) - \left( \sum_{k = 0}^{n} k \frac{p_n (k)}{n!} \right) &= 0 \\
        \iff \left( \sum_{k = 0}^{n + 1} \frac{k}{(n + 1)!} \binom{n + 1}{k} D_{n + 1 - k} \right) - \left( \sum_{k = 0}^{n} \frac{k}{n!} \binom{n}{k} D_{n - k} \right) &= 0
    .\end{align*}
    From here, we can call upon a well-known \MarginComment{This is the only
    part that I feel is a little bit cheating, but in my defense it is a
    decently well-known expression and it can be proven using inclusion-exclusion.}
    closed form for the number of derangements of \( m \) items:
    \[
        D_m = m! \sum_{i = 0}^{m} \frac{(-1)^i}{i!}
    .\]
    From this, we have that \( f(n+1) - f(n) \) is
    \begin{align*}
        &\left( \sum_{k = 0}^{n + 1} \frac{k}{k!} \sum_{i = 0}^{n + 1 - k} \frac{(-1)^i}{i!} \right) - \left( \sum_{k = 0}^{n} \frac{k}{k!} \sum_{i = 0}^{n - k} \frac{(-1)^i}{i!} \right) \\
        &= \frac{1}{n!} + \sum_{k = 0}^{n} \frac{k}{k!} \cdot \frac{(-1)^{n - k + 1}}{(n - k + 1)!} \\
        &= \frac{1}{n!} + \sum_{k = 1}^{n} \frac{1}{(k - 1)! \left( n - (k - 1) \right)!} (-1)^{n - (k - 1)} \\
        &= \frac{1}{n!} + \sum_{k = 0}^{n - 1} \frac{1}{k! (n - k)!} (-1)^{n - k} \\
        &= \sum_{k = 0}^{n} \frac{1}{k! (n - k)!} (-1)^{n - k} \\
        &= \frac{1}{n!} \sum_{k = 0}^{n} \frac{n!}{k! (n - k)!} (-1)^{n - k} \\
        &= \frac{1}{n!} \sum_{k = 0}^{n} \binom{n}{k} (-1)^{n - k}
    .\end{align*}
    By the binomial theorem, this is precisely \( (1 + (-1))^n / n! = 0 \).
    This completes the inductive step and proves that \( f(n) = 1 \) for all \(
    n \geqslant 1 \), which is equivalent to our original problem.
\end{proof}

