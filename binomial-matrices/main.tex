\documentclass[a4paper, 12pt]{article}

\usepackage{chirpstyle}

\begin{document}

\section*{Binomial Matrices}

\begin{sidebox}
    \begin{idea}
        Consider \( (n + 1) \times (n + 1) \) matrices of the form \( A_{ij} = \binom{i - 1}{j - 1} \), with the convention that \( \binom{i}{j} = 0 \) for \( j > i \). Denote these matrices by \( B_n \). What kind of properties do these matrices (and perhaps similar ones) have?
    \end{idea}
\end{sidebox}

\begin{example}
    The first few such matrices are
    \[
        B_0 = \begin{bmatrix}
            1
        \end{bmatrix},
        B_1 = \begin{bmatrix}
            1 & 0 \\
            1 & 1
        \end{bmatrix},
        B_2 = \begin{bmatrix}
            1 & 0 & 0 \\
            1 & 1 & 0 \\
            1 & 2 & 1
        \end{bmatrix},
        B_3 = \begin{bmatrix}
            1 & 0 & 0 & 0 \\
            1 & 1 & 0 & 0 \\
            1 & 2 & 1 & 0 \\
            1 & 3 & 3 & 1
        \end{bmatrix}
    .\]
\end{example}

\begin{observation}
    Immediately we can see that by definition such matrices are lower unitriangular, as for any such matrix \( A \), \( A_{ij} = 0 \) for \( j > i \) and \( A_{ii} = \binom{i - 1}{i - 1} = 1 \). This is also obvious from the examples.

    With this structure comes a few properties:
    \begin{itemize}
        \item \( \det B_n = 1 \).
        \item \( \det (B_n - \lambda I) = (\lambda - 1)^{n + 1} \).
        \item \( B_n \) is not diagonalizable.
    \end{itemize}
\end{observation}

\begin{question}
    What is \( B_n B_n^T \)?
\end{question}

\begin{idea}
    Using Mathematica, we can gain the intuition that the answer should be \( (B_n B_n^T)_{ij} = \binom{i + j - 2}{i - 1} = \binom{i + j - 2}{j - 1} \). We can prove this is true by observing that
    \begin{align*}
        (B_n B_n^T)_{ij} &= \sum_{k = 0}^{n} (B_n)_{ik} (B_n^T)_{kj} \\
        &= \sum_{k = 1}^{n} \binom{i - 1}{k - 1} \binom{j - 1}{k - 1} \\
        &= \sum_{k = 0}^{\min(i - 1, j - 1)} \binom{i - 1}{k} \binom{j - 1}{j - 1 - k} 
    .\end{align*}
    If we WLOG let \( i \ge j \), then Vandermonde's identity tells us that \( (B_n B_n^T)_{ij} \) indeed is \( \binom{i + j - 2}{j - 1}  \).
\end{idea}

\begin{question}
    What is the inverse of \( B_n \)?
\end{question}

\begin{idea}
    Since \( B_n \) is unitriangular, it's very easy to work out some small examples, so let us do so. For \( B_3 \), we have
    \begin{align*}
        \begin{bmatrix}
            1 & 0 & 0 & 0 & \bigm| & b_1 \\
            1 & 1 & 0 & 0 & \bigm| & b_2 \\
            1 & 2 & 1 & 0 & \bigm| & b_3 \\
            1 & 3 & 3 & 1 & \bigm| & b_4
        \end{bmatrix} &\longrightarrow
        \begin{bmatrix}
            1 & 0 & 0 & 0 & \bigm| & b_1 \\
            0 & 1 & 0 & 0 & \bigm| & b_2 - b_1 \\
            1 & 2 & 1 & 0 & \bigm| & b_3 \\
            1 & 3 & 3 & 1 & \bigm| & b_4
        \end{bmatrix} \\
        &\longrightarrow 
        \begin{bmatrix}
            1 & 0 & 0 & 0 & \bigm| & b_1 \\
            0 & 1 & 0 & 0 & \bigm| & b_2 - b_1 \\
            0 & 0 & 1 & 0 & \bigm| & b_3 - 2b_2 + b_1 \\
            1 & 3 & 3 & 1 & \bigm| & b_4
        \end{bmatrix} \\
        &\longrightarrow 
        \begin{bmatrix}
            1 & 0 & 0 & 0 & \bigm| & b_1 \\
            0 & 1 & 0 & 0 & \bigm| & b_2 - b_1 \\
            0 & 0 & 1 & 0 & \bigm| & b_3 - 2b_2 + b_1 \\
            0 & 0 & 0 & 1 & \bigm| & b_4 - 3b_3 + 3b_2 - b_1
        \end{bmatrix}
    ,\end{align*}
    which I suppose is somewhat obvious. This tells us that
    \[
        B_3^{-1} = \begin{bmatrix}
            1 & 0 & 0 & 0 \\
            -1 & 1 & 0 & 0 \\
            1 & -2 & 1 & 0 \\
            -1 & 3 & -3 & 1
        \end{bmatrix}
    ,\]
    which allows one to guess that \( (B_n^{-1})_{ij} = (-1)^{i + j} (B_n)_{ij} = (-1)^{i + j} \binom{i - 1}{j - 1} \). Let's try and prove this. We have that
    \[
        (B_n B_n^{-1})_{ij} = \delta_{ij}
    ,\]
    so we must have that
    \begin{align*}
        (B_n B_n^{-1})_{ij} &= \sum_{k = 1}^{n+1} (B_n)_{ik} (B_n^{-1})_{kj} \\
        &= \sum_{k = 1}^{n + 1} (-1)^{k + j} \binom{i - 1}{k - 1} \binom{k - 1}{j - 1} \\
        &= \sum_{k = 0}^{n} (-1)^{k + 1 + j} \binom{i - 1}{k} \binom{k}{j - 1}  \\
        &= \textcolor{WildStrawberry}{\textsf{TODO}}
    .\end{align*}
    % Hmmmm
    % https://artofproblemsolving.com/community/c7h1973655p13689198
    % https://approach0.xyz/search/?q=OR%20content%3A%24%5Csum_k%20%5Cleft(-1%5Cright)%5Ek%5Cbinom%7Ba%7D%7Bk%7D%5Cbinom%7Bk%7D%7Bb%7D%24&p=1
\end{idea}

\begin{remark}
    Knowing the inverse actually allows one to calculate a formula for the number of derangements of \( n \) objects without using the inclusion-exclusion principle, denoted by \( D_n \). In particular,
    \[
        \begin{bmatrix}
            D_0 \\
            D_1 \\
            \vdots \\
            D_{n}
        \end{bmatrix} =
        B_{n}^{-1} \begin{bmatrix}
            0! \\
            1! \\
            \vdots \\
            n!
        \end{bmatrix}
    ,\]
    which shows that
    \begin{align*}
        D_n &= \sum_{k = 0}^{n} (-1)^k \binom{n}{n - k} (n-k)! \\
        &= \sum_{k = 0}^{n} (-1)^k \frac{n!}{k!} \\
        &= n! \sum_{k = 0}^{n} \frac{(-1)^k}{k!}
    .\end{align*}
\end{remark}

\end{document}
