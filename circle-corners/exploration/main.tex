\documentclass{article}

\usepackage{amsmath}
\usepackage{amsthm}
\usepackage{amssymb}

\usepackage[margin=1.8cm]{geometry}
\usepackage{tikz}

\begin{document}

\section*{Convergent Circle Corners}

\subsection*{Circle Areas}

Given two points on the first quadrant of unit circle \( A, B \), we may define
a new unique point \( C = A \oplus B \) through the following: extend a
horizontal line through the \( y \)-coordinate of the point with the smaller \(
x \)-coordinate. Next, extend a vertical line through the \( x \)-coordinate of
the point with the larger \( x \)-coordinate. Find the point of intersection
between these two lines and draw a line with slope \( 1 \) running through this
point. The new point \( C \) shall be the intersection of this line with the
unit circle in the first quadrant.

More explicitly, suppose that \( A = \left( x_1, y_1 \right), B = \left( x_2,
y_2 \right), x_1 \leqslant x_2 \). Then we have that
\[
    C = \left( \frac{x_2 - y_1 + \sqrt{2 - \left( x_2 - y_1 \right)^2}}{2}, \frac{y_1 - x_2 + \sqrt{2 - \left( y_1 - x_2 \right)^2}}{2} \right)
.\]
Intuitively, we also have that \( A = A \oplus A \).

% Lol the figure looks kinda wonky but it'll do for now I guess
\begin{center}
\begin{tikzpicture}[scale=4]
    \draw (1, 0) arc (0:90:1);
    \draw[->] (-0.05, 0) -- (1.2, 0);
    \draw[->] (0, -0.05) -- (0, 1.2);

    \node (A) at ({cos(75)}, {sin(75)}) {};
    \node[anchor=south] at (A) {A};
    \fill (A) circle (0.012);

    \node (B) at ({cos(30)}, {sin(30)}) {};
    \node[anchor=west] at (B) {B};
    \fill (B) circle (0.012);

    \draw[->] ({cos(75)}, {sin(75)}) -- ({cos(75) + 1}, {sin(75)});
    \draw[->] ({cos(30)}, {sin(30)}) -- ({cos(30)}, {sin(30) + 0.7});
    \draw[->] ({cos(30) + 0.02}, {sin(75) + 0.02}) -- ({cos(30) - 0.5}, {sin(75) - 0.5});

    \node (C) at (0.655390115, 0.75529053) {};
    \node[anchor=north] at (C) {C};
    \fill (C) circle (0.012);
\end{tikzpicture}
\end{center}

With this, we can begin to build up our circle corner converging series. We can
build this recursively, starting with an initial condition. For \( n \geqslant
0 \), define \( S_n \) to be the ordered set (according to \( x \)-coordinate)
of all corner points lying on the unit circle quadrant. We let \( S_0 = \left(
\left( 0, 1 \right), \left( 1, 0 \right) \right) \) to start with and then use
the following recursive definition: suppose \( S_n = \left( P_1, P_2, \ldots,
P_k \right) \); we have
\[
    S_{n+1} = \left( P_1, P_1 \oplus P_2, P_2, \ldots, P_{k-1}, P_{k - 1} \oplus P_k, P_k \right)
.\]
% Ok I fixed it
% (Actually, I'm now doubting that this is in fact true, considering that we only
% really want to find the corners between neighboring points and not all of
% them. The fact that the sum seems to move towards \( \pi \) for increasing
% values of \( n \) isn't really too much indication that things are going right
% as the Darboux integral of any decently well-behaved partition of \( \left[ 0,
% 1 \right] \) will likely approach the Riemann integral, and give us our value
% of \( \pi \). Certainly the operation we have is correct, but I don't think
% building the set recursively is as easy as this).
One can confirm that for \( n \geqslant 1 \), there are \( k = 2^n + 1 \) points in
this ordered set.

We now desire the area of the converging corners shape. Fix a specific \( n \)
and look at the ordered set of points \( S_n = \left( \left( x_1, y_1 \right),
\left( x_2, y_2 \right), \ldots, \left( x_k, y_k \right) \right) \). We then
have that the total area is
\[
    A_n = 4 \sum_{i = 1}^{k - 1} y_i \left( x_{i + 1} - x_i \right)
.\]
We assert that this area approaches \( \pi \) as \( n \to \infty \), but I
can't properly show this right now.

Now that we have mathematically formulated this, perhaps it would be a good
exercise to code it. As for how to formally show that this approaches \( \pi
\), I'll have to think about it (and perhaps try a different approach). For
now, it will do good to verify it empirically though. Certainly the previous
series looks suspiciously like an integral (in fact it is called the Darboux
integral), so perhaps our proof shall rely upon this.

In coding a rough example, we can see empirically that the area does roughly
approach \( \pi \), albeit a bit slowly considering the number of points. We have
\begin{center}
\begin{tabular}{c|c|c}
    \( n \) & Points & Area \\
    \hline
    1 & 3 & 3.6568542494923806\\
    2 & 5 &  3.447477652576853 \\
    3 & 9 &  3.324791968290031 \\
    4 & 17 & 3.2536764558450173 \\
    5 & 33 & 3.2121295100035105 \\
    \vdots & \vdots & \vdots \\
    20 & 1048577 & 3.1430271877389186 
\end{tabular}
\end{center}
I'm not quite sure how to express a good closed form for any of this, but
certainly we can prove that the maximum difference between the \( x \)
coordinates of the points goes to \( 0 \), but I'm not sure whether this is any
help for proving that the Darboux integral approaches the Riemann integral or
anything of the sort.

\end{document}
