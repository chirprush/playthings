\documentclass[a4paper, 12pt]{article}

\usepackage{chirpstyle}

\begin{document}

\section*{The Euler Totient is Even}

After pondering about some stuff relating to modular matrices from a while back, I thought of a very quick, and easy to derive way of showing that the Euler totient function \( \phi (n) \) is even (except in the case of \( n = 2 \)). We'll first start with a review.

\begin{sidebox}
    \begin{definition}
        The \textbf{Euler totient function} \( \phi(n) \) is defined to be the number of natural numbers less than \( n \) that are relatively prime to \( n \) (that is \( \gcd(n, k) = 1 \) for the natural \( k \)).
    \end{definition}
\end{sidebox}

While there are definitely number theory proofs of this always being even, we shall show this is true with group theory.

Let \( A_n \) denote the set, of natural numbers less than \( n \) that are relatively prime to \( n \). We assert that this set forms a group when equipped with multiplication modulo \( n \), and we shall denote this group by \( \mathbb{Z}_n^\times \).

\begin{sidebox}
    \begin{theorem}
        \( \mathbb{Z}_n^\times \) forms a group.
    \end{theorem}
\end{sidebox}

\begin{proof}
    We shall prove the three necessary conditions: associativity, the existence of an identity, and inverses for each element.
    \begin{itemize}
        \item \textbf{Associativity}: This one is obvious, as multiplication is associative and even commutative.
        \item \textbf{Identity element}: This condition is also obvious, as one can verify that \( 1 \) is in \( \mathbb{Z}_n^\times \) and satisfies the conditions to be the identity.
        \item \textbf{Inverse elements}: By the Extended Euclidean algorithm, we know that for any \( a \in \mathbb{Z}_n^\times \), \( a^{-1} \) exists in general. Thus, it suffices to prove that this \( a^{-1} \) is in \( \mathbb{Z}_n^\times \); that is, we must prove that \( \gcd(a^{-1}, n) = 1 \).

            Observe that for any \( k \), \( \gcd(ka^{-1}, n) \ge \gcd(a^{-1}, n) \). If we take \( k = a \), however, we see that \( \gcd(ka^{-1}, n) = \gcd(1, n) = 1 \), so clearly \( \gcd(a^{-1}, n) = 1 \).
    \end{itemize}
\end{proof}

This group has a direct correlation with the original problem at hand. Indeed, \( |\mathbb{Z}_n^\times| = |A_n| = \phi (n) \). Now that we have a group structure over the set, however, we can use some fun group theory concepts. In particular, recall Lagrange's theorem, or a certain version of it at least.

\begin{sidebox}
    \begin{theorem}[Lagrange's theorem]
        For any subgroup \( H \) of some group \( G \), \( |H| \) divides \( |G| \).
    \end{theorem}
\end{sidebox}

Using this we can trivially show that the Euler totient is even.

\begin{sidebox}
    \begin{theorem}
        \( \phi(n) \) is even for \( n \ge 3 \).
    \end{theorem}
\end{sidebox}

\begin{proof}
    We have that \( \mathbb{Z}_n^\times \) is a group and \( |\mathbb{Z}_n^\times| = \phi(n) \). Observe, however, that \( \{1, -1\} \subseteq \mathbb{Z}_n^\times \) is a subgroup of this group, and it clearly has order \( 2 \). Thus, we must have that \( 2 \mid \phi(n) \).
\end{proof}

The only exception to this is the case of \( n = 2 \), where \( \phi(2) = 1 \). The reason that this argument fails in this case is because the subgroup is actually of order \( 1 \), as \( 1 \) is congruent to \( -1 \) modulo \( 2 \).

\end{document}
