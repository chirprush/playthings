\documentclass[a4paper, 12pt]{article}

\usepackage{chirpstyle}

\begin{document}

\section*{Matrices Over Modular Arithmetic}

Consider matrices from the set \( M_{k \times k}(\mathbb{Z}_m) \), where \( \mathbb{Z}_m \) denotes the field of integers modulo \( m \). We wish to explore the properties of such matrices, especially when taken to integer powers. For shorthand, we call such a matrix a \textit{modular matrix}.

\begin{example}
    One such matrix is
    \[
        \begin{bmatrix}
            1 & 3 & 2 \\
            0 & 6 & 5 \\
            4 & 4 & 2
        \end{bmatrix} \in M_{3 \times 3}(\mathbb{Z}_{7})
    .\]
\end{example}

\begin{theorem}
    Let \( A \) be any modular matrix of dimension \( k \) and modulus \( m \). The sequence \( A, A^2, A^3, \ldots \) is eventually periodic.
\end{theorem}

\begin{proof}
    The sequence being periodic is equivalent to any matrix appearing more than once in the sequence. Since the sequence is infinite and there are \( m^{k^2} \) modular matrices of same dimension and modulus, we can always choose a sequence index greater than \( m^{k^2} \) and argue by the Pigeonhole Principle that there must exist a duplicate matrix. Thus, the sequence must repeat at some point (not necessarily with the starting matrix).
\end{proof}

With this in mind, we now begin discussing invertibility.

\begin{definition}
    The \textit{determinant} of a modular matrix with modulus \( m \) is simplify defined to be the regular determinant taken modulo \( m \).
\end{definition}

\begin{theorem}
    A modular matrix \( A \in M_{k \times k}(\mathbb{Z}_m) \) is invertible iff \( \det A \) is invertible in \( \mathbb{Z}_m \). That is to say, iff \( \gcd{(\det A, m)} = 1 \).
\end{theorem}

\begin{proof}
    Since the properties of determinants still apply, \( \det A^{-1} = (\det A)^{-1} \), so clearly if \( \gcd{(\det A, m)} > 1 \), this cannot exist. Thus, it suffices to show that all matrices with \( \det A \) such that \( \gcd(\det A, m) = 1 \) are invertible.

    \textcolor{WildStrawberry}{TODO.}
\end{proof}

\begin{notation}
    We denote the set of invertible modular matrices with size \( k \times k \) and modulo \( m \) by \( N_{k \times k}(\mathbb{Z}_m) \).
\end{notation}

With this, we can now see some sets of modular matrices have a structure to them.

\begin{theorem}
    \( N_{k \times k}(\mathbb{Z}_{m}) \) forms a group under multiplication.
\end{theorem}

\begin{proof}
    Most of the necessary properties follow readily from matrix algebra, so we may quickly verify the conditions:
    \begin{itemize}
        \item \textbf{Associativity}: We have trivially that \( A(BC) = (AB)C \).
        \item \textbf{Identity Element}: The unique identity element for \( M_{k \times k}(\mathbb{Z}_m) \) is \( I_k \).
        \item \textbf{Inverse Element}: It is already stated that each element has a (two-sided) inverse. Since this inverse is also invertible, it is included the group.
    \end{itemize}
\end{proof}

Because this group is of finite order, we have that for any invertible modular matrix \( A \), there exists some smallest non-zero exponent \( r \), the order of the element, such that \( A^r = I_k \). We wish to the order, or some multiple of it, which shall give us a tool to reduce the exponent of a modular matrix power.

This motivates us to take a look at the cyclic group generated by the element \( A \). Note that for any group \( G \) where \( | G | = n \) we have that \( g^n = e \) for all \( g \in G \), so we are motivated to find the order of this cyclic group, or some multiple of it.

Observe that the cyclic group generated by some invertible modular matrix \( A \) is a subgroup of all invertible modular matrices of same size and modulus. By Lagrange's theorem, we can find that the order of all such invertible matrices is a multiple of the order of \( A \). Thus, our focus now shifts to the entire group of invertible matrices and finding its order.

\begin{remark}
    It should be noted that there is also a well-behaved closed form for the number of invertible modular matrices for \textit{prime} \( m \). Consider an argument where we choose \( k \) linearly independent vectors of size \( k \) one-by-one. For the first vector, we have \( m^k - 1 \) choices (all but the zero vector). For the second vector, we have \( m^k - m \) choices since there are \( m \) vectors that can be formed as a linear combination of the first chosen vector. The third has \( m^k - m^2 \) choices by a similar reasoning. In general, the number of invertible matrices for a prime modulus \( m \) is given by
    \[
        (m^k - 1)(m^k - m)(m^k - m^2) \cdots (m^k - m^{k - 1}) = \prod_{i = 0}^{k - 1} (m^k - m^i)
    .\]
    The reason this fails to count all invertible matrices for composite \( m \) is because not all matrices with nonzero determinant are invertible in such instances.
\end{remark}

Since this doesn't generalize well for prime powers, we'll have to find some other approach. It suffices to find a multiple of the order for prime powers, since for any composite number, we may simply take the LCM of the orders of its pure prime power components. This motivates a new approach.

\begin{definition}
    Let the \textit{determinant frequency} of some congruency class \( a \) with respect to the set \( N_{k \times k}(\mathbb{Z}_m) \) be the number of matrices \( A \in N_{k \times k}(\mathbb{Z}_m) \) such that \( \det A = a \). We shall denote this frequency by \( \mathfrak{q} (a) \).
\end{definition}

With this, we can decompose the invertible matrices into a sum of over all determinants that give inverses. In particular, we want to calculate
\[
    \sum_{\gcd(a, m) = 1} \mathfrak{q} (a)
.\]
While this doesn't do much for us, there's one really nice simplification that we can make.
\begin{theorem}
    For all \( a \) such that \( \gcd(a, m) = 1 \), \( \mathfrak{q}(a) \) are all equal.
\end{theorem}

\begin{proof}
    Consider the mapping \( x \mapsto qx \), where \( \gcd(a, m) = 1 \). Since \( q^{-1} \) exists, this mapping is invertible, and is thus a bijection over \( \mathbb{Z}_m \). We observe that applying such a mapping elementwise to a modular matrix also forms a bijection. Since the properties of determinants apply, this mapping transforms \( \det A \) to be \( q^k \det A \). Since \( \gcd(\det A, m) = \gcd(a, m) = 1 \) and \( \gcd(q, m) = 1 \), we can always choose a \( q \) such that \( q^k = a^{-1} \). This means that we can always construct a bijection between modular matrices of determinant \( a \) and modular matrices of determinant \( 1 \), which suffices to show that each of the specified congruency classes for determinants contains the same number of elements.
\end{proof}
This allows us to simplify our sum a bit further, transforming into
\[
    \sum_{\gcd(a, m) = 1} \mathfrak{q} (a) = \varphi(m) \mathfrak{q}(1)
.\]
Unfortunately, I'm not so sure how hard it is to determine \( \mathfrak{q}(1) \) in general. Of course the set of modular matrices of determinant \( 1 \) forms a group in of itself, so there's something to work with, but I'm not sure how far one can get in terms of counting. Perhaps we should aim for a general asymptotic, although this isn't really applicable to reducing matrix powers.

Perhaps I'll come back to this later if I have any ideas.

\end{document}
