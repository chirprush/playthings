\documentclass{exam}

\usepackage[utf8]{inputenc}
\usepackage[OT1]{fontenc}
\usepackage[margin=2cm, tmargin=1.5cm, rmargin=5cm]{geometry}

\usepackage{amsmath}
\usepackage{amssymb}
\usepackage{amsthm}

\usepackage{newpxtext}
\usepackage{newpxmath}

\renewcommand{\questionlabel}{\textbf{1.\thequestion.}}

\pagestyle{empty}

\begin{document}

\textbf{Modern Algebra: 1.1 Exercises}

\begin{questions}
    \question Let \( S = \{ w, x, y, z \} \) and \( T = \{ 1, 2, 3, 4 \} \), and define \( \upalpha \colon S \to T \) and \( \upbeta \colon S \to T \) by \( \upalpha \left( w \right) = 2 \), \( \upalpha \left( x \right) = 4 \), \( \upalpha \left( y \right) = 1 \), \( \upalpha \left( z \right) = 2 \) and \( \upbeta \left( w \right) = 4 \), \( \upbeta \left( x \right) = 2 \), \( \upbeta \left( y \right) = 3 \), \( \upbeta \left( z \right) = 1 \).

    \begin{parts}
        \part Is \( \upalpha \) one-to-one? Is \( \upbeta \) one-to-one? Is \( \upalpha \) onto? Is \( \upbeta \) onto?
        \part Let \( A = \{ w, y \} \) and \( B = \{ x, y, z \} \). Determine each of the following subsets of \( T \): \( \upalpha \left( A \right) \), \( \upbeta \left( B \right) \), \( \upalpha \left( A \cap B \right) \), \( \upbeta \left( A \cup B \right) \).
    \end{parts}

    \vspace{1.2cm}

    \question Let \( \upalpha \), \( \upbeta \), and \( \upgamma \) be mappings from \( \vvmathbb{Z} \) to \( \vvmathbb{Z} \) defined by \( \upalpha \left( n \right) = 2n \), \( \upbeta \left( n \right) = n + 1 \), and \( \upgamma \left( n \right) = n^3 \) for each \( n \in \vvmathbb{Z} \).

    \begin{parts}
        \part Which of the three mappings are onto?
        \part Which of the three mappings are one-to-one?
        \part Determine \( \upalpha \left( \vvmathbb{N} \right) \), \( \upbeta \left( \vvmathbb{N} \right) \), and \( \upgamma \left( \vvmathbb{N} \right) \).
    \end{parts}

    \vspace{1.2cm}
    
    \textit{For Problems 1.3-1.6, assume \( S = \{ x, y, z \} \) and \( T = \{ 1, 2, 3 \} \).}

    \question How many mappings are there from \( S \) to \( T \)?

    \vspace{0.5cm}

    \question How many mappings are there from \( S \) onto \( T \)?

    \vspace{0.5cm}

    \question How many one-to-one mappings are there from \( S \) to \( T \)?

    \vspace{0.5cm}

    \question How many mappings are there from \( S \) to \( \{ 1, 2 \} \)?

    \vspace{0.6cm}

    \textit{For Problems 1.7-1.10, assume that \( S \) and \( T \) are sets, \( \upalpha \colon S \to T \), and \( \upbeta \colon S \to T \). Complete each of the following statements. (The discussion of quantifiers in Appendix B may help.)}

    \question \( \upalpha \) is not onto iff for some....
    \question \( \upalpha \) is not one-to-one iff....
    \question \( \upalpha \ne \upbeta \) iff....
    \question \( \upbeta \) is one-to-one and onto iff for each \( y \in T \)....

    \vspace{0.6cm}

    \textit{Each \( f \) in Problems 1.11-1.16 defines a mapping from \( \vvmathbb{R} \) (or a subset of \( \vvmathbb{R} \)) to \( \vvmathbb{R} \). Determine which of these mappings are onto and which are one-to-one. Also describe \( f \left( P \right) \) in each case, for \( P \) the set of positive real numbers.}

    \question \( f \left( x \right) = 2x \)
    \question \( f \left( x \right) = x - 4 \)
    \question \( f \left( x \right) = x^3 \)
    \question \( f \left( x \right) = x^2 + x \)
    \question \( f \left( x \right) = e^x \)
    \question \( f \left( x \right) = \tan{x} \)

    \vspace{0.6cm}

    \newpage

    \textit{In Problems 1.17 and 1.18 let \( A \) denote the set of odd natural numbers, \( B \) the set of even natural numbers, and \( C \) the set of natural numbers that are multiples of \( 4 \).}

    \question With \( \upalpha \colon \vvmathbb{N} \to \vvmathbb{N} \) as given below, describe \( \upalpha \left( A \right) \), \( \upalpha \left( B \right) \), and \( \upalpha \left( C \right) \).

    \[
        \upalpha \left( n \right) = 2n
    \]

    \question With \( \upbeta \colon \vvmathbb{N} \to \vvmathbb{N} \) as given below, describe \( \upbeta \left( A \right) \), \( \upbeta \left( B \right) \), and \( \upbeta \left( C \right) \).

    \[
        \upbeta \left( n \right) = \begin{cases}
            \left( n + 1 \right)/2 & \text{if } n \text{ is odd} \\
            n/2 & \text{if } n \text{ is even}
        \end{cases}
    \]

    \vspace{0.6cm}

    \textit{In Problems 1.19 and 1.20, for each \( n \in \vvmathbb{Z} \) the mapping \( f_n \colon \vvmathbb{Z} \to \vvmathbb{Z} \) is defined by \( f_n \left( x \right) = nx \).}

    \question For which values of \( n \) is \( f_n \) onto?
    \question For which values of \( n \) is \( f_n \) one-to-one?

    \vspace{0.6cm}

    \question Assume that \( S \) and \( T \) are finite sets containing \( m \) and \( n \) elements, respectively.

    \begin{parts}
        \part How many mappings are there from \( S \) to \( T \)?
        \part How many one-to-one mappings are there from \( S \) to \( T \)? (Consider two cases: \( m > n \) and \( m \le n \).)
    \end{parts}

    \vspace{0.3cm}

    \question (a) How many mappings are there from a 2-element set onto a 2-element set? (b) from a 3-element set onto a 2-element set? (c) from an \( n \)-element set onto a 2-element set?

    \vspace{0.3cm}

    \question A mapping \( f \colon \vvmathbb{R} \to \vvmathbb{R} \) is onto iff each horizontal line (line parallel to the \( x \)-axis) intersects the graph of \( f \) at least once.

    \begin{parts}
        \part Formulate a similar condition for \( f \colon \vvmathbb{R} \to \vvmathbb{R} \)  to be one-to-one.
        \part Formulate a similar condition for \( f \colon \vvmathbb{R} \to \vvmathbb{R} \)  to be both one-to-one and onto.
    \end{parts}

    \vspace{0.3cm}

    \question For each order pair \( \left( a, b \right) \) of integers define a mapping \( \upalpha_{a,b} \colon \vvmathbb{Z} \to \vvmathbb{Z} \) by \( \upalpha_{a,b} \left( n \right) = an + b \).

    \begin{parts}
        \part For which pairs \( \left( a, b \right) \), is \( \upalpha_{a,b} \) onto?
        \part For which pairs \( \left( a, b \right) \), is \( \upalpha_{a,b} \) one-to-one?
    \end{parts}

    \vspace{0.3cm}

    \question With \( \upbeta \) as defined in Problem 1.18, for each \( n \in \vvmathbb{N} \) the equation \( \upbeta \left( x \right) = n \) has exactly two solutions. (The solutions of \( \upbeta \left( x \right) = 2 \) are \( x = 3 \) and \( x = 4 \), for example.)

    \begin{parts}
        \part Define a mapping \( \upgamma \colon \vvmathbb{N} \to \vvmathbb{N} \) such that for each \( n \in \vvmathbb{N} \) the equation \( \upgamma \left( x \right) = n \) has exactly three solutions.
        \part Define a mapping \( \upgamma \colon \vvmathbb{N} \to \vvmathbb{N} \) such that for each \( n \in \vvmathbb{N} \) the equation \( \upgamma \left( x \right) = n \) has exactly \( n \) solutions. (It suffices to describe \( \upgamma \) in words.)
        \part Define a mapping \( \upgamma \colon \vvmathbb{N} \to \vvmathbb{N} \) such that for each \( n \in \vvmathbb{N} \) the equation \( \upgamma \left( x \right) = n \) has infinitely many solutions.
    \end{parts}

    \vspace{0.3cm}

    \question Prove that there is a mapping from a set to itself that is one-to-one but not onto if there is a mapping from the set to itself that is onto but not one-to-one. (Compare Problems 1.17 and 1.18).

    \vspace{0.3cm}

    \question Prove that if \( \upalpha \colon S \to T \), and \( A \) and \( B \) are subsets of \( S \), then \( \upalpha \left( A \cup B \right) = \upalpha \left( A \right) \cup \upalpha \left( B \right) \).

    \vspace{0.3cm}

    \question

    \begin{parts}
        \part Prove that if \( \upalpha \colon S \to T \), and \( A \) and \( B \) are subsets of \( S \), then \( \upalpha \left( A \cap B \right) \subseteq \upalpha \left( A \right) \cap \upalpha \left( B \right) \).
        \part Give an example (specific \( S \), \( T \), \( A \), \( B \), and \( \upalpha \)) to show that equality need not hold in part (a). (For the simplest examples \( S \) will have two elements.)
    \end{parts}

    \vspace{0.3cm}

    \question Prove that a mapping \( \upalpha \colon S \to T \) is one-to-one iff \( \upalpha \left( A \cap B \right) = \upalpha \left( A \right) \cap \upalpha \left( B \right) \) for every pair of subsets \( A \) and \( B \) of S. (Compare Problem 1.28.)

\end{questions}

\end{document}
