\documentclass[a5paper]{article}

\usepackage{chirpstyle}

\begin{document}

\section*{Having Some Fun With Pell Equations}

Pell equations are quadratic Diophantine equations of a special kind with
infinite solutions that are very useful in a wide range of settings. In
particular, though, they seem to come up in Project Euler problems quite a bit,
so I figured I might as well document some of the useful relations for
computation and such because I seem to forget them everytime.

\subsection*{Overview}

The simplest Pell equations are of the form
\[
    x^2 - Dy^2 = 1
,\]
where \( x \) and \( y \) are positive integers and we fix \( D \) to be a non-square value. We shall call
these \textit{fundamental}. This being said, we may also encounter \textit{generalized Pell equations} of the form
\[
    x^2 - Dy^2 = k
.\]
The solutions to these are quite easily derived from the corresponding
fundamental Pell equation, though.

The basic idea behind Pell equations is that if we have the smallest, trivial solution
(i.e. the smallest \( (x, y) \) values that satisfy the equation), then we can
generate all infinite solutions of the Pell equation using some computation.
While I don't know of a way to find the trivial solutions, once we do have
them, everything goes quite swimmingly and nice.

\subsection*{Basic Results}

One of the first things to show (which is very easy) is that Pell equations
don't have solutions if \( D \) is a perfect square.
\begin{chirpbox}
    \begin{theorem}
        There exists no solutions \( (x, y) \) to the equation
        \[
            x^2 - Dy^2 = 1
        \]
        when \( D = n^2 \) for some natural number \( n \).
    \end{theorem}
\end{chirpbox}
\begin{proof}
    This follows trivially from factoring the difference of squares:
    \[
        x^2 - n^2 y^2 = (x - ny)(x + ny) = 1
    .\]
    By the fundamental theorem of arithmetic, this only holds true if both
    terms are either \( 1 \) or \( -1 \), but this can only be the case if \(
    (x, y) = (\pm1, 0) \). As we've stated before, though, \( x,y \) must be
    positive, so we may discard this and say that there are no solutions.
\end{proof}
The main tenet of Pell equations is that, once we have a trivial solution, we
can generate all other solutions. This is based on the following fact:
\begin{chirpbox}
    \begin{theorem}[Brahmagupta's Identity]
        We have that
        \[
            (x^2 - D y^2)(w^2 - Dz^2) = (xw + Dyz)^2 - D(xz + yw)^2
        .\]
    \end{theorem}
\end{chirpbox}
The proof is somewhat just algebra and shall be omitted, but what's important
here to recognize is that multiplying two Pell solutions gives us another Pell
solution. Particularly for fundamental Pell equations, this gives us a way to
generate all\footnote{This does not prove that all solutions are reached, but
just trust that all are lmao :thumbsup:} solutions because on the right hand
side we have that \( 1 \cdot 1 = 1 \). This also bridges the gap between the
fundamental Pell equations and the generalized ones. In particular, if we
multiply a trivial solution of the generalized Pell equation by the trivial
solution of the corresponding (that is, one with the same \( D \), but \( k = 1
\)) fundamental equation, we get another solution of the generalized Pell
equation.

\subsection*{Examples}

\end{document}
