\documentclass[a4paper, 12pt]{article}

\usepackage{chirpstyle}

\begin{document}

\section*{Custom Color Filters}

\begin{blackbox}
    \begin{problem}
        Suppose I have a list of colors that form a colorscheme. How can I
        create a (relatively good-looking) filter that respects this
        colorscheme?
    \end{problem}
\end{blackbox}

We start with a few simplifying ideas.

\begin{idea}
    We don't care about neighboring pixels, so we're just transforming
    individual pixels. Formally, let \( C := [0, 1]^3 \) be our color space,
    with the vector components corresponding to intensities of red, green, and
    blue. Our filter consists entirely of some well chosen, continuous function
    \( f : C \to C \).
\end{idea}

\begin{idea}
    There's a nice physical intuition for what we're trying to achieve. The
    colors in our colorscheme turn into points of \( C \), which we can denote
    by the set \( P \). In trying to transform pixels to better respect the
    colorscheme, we should ideally have some notion of moving points in a
    neighborhood of \( P \) closer to \( P \), almost as if the points in \( P
    \) have their own gravity.
\end{idea}

With these two ideas, it suffices to just put some numbers and basic function
forms to create a cool color filter.

Ideally, the attraction of any point in a colorscheme should be radially symmetric and pointing inwards. Thus, we can further classify \( f \) to be of the form
\[
    f(\vec{x}) = \vec{x} + \sum_{\vec{p} \in P} s(\| \vec{p} - \vec{x} \|) \cdot \nabla (\| \vec{p} - \vec{x} \|)
,\]
but since
\[
    \nabla (\| \vec{p} - \vec{x} \|) = \frac{2(\vec{p} - \vec{x})}{\| \vec{p} - \vec{x} \|}
,\]
we can write \( f \) as
\[
    f(\vec{x}) = \vec{x} +  \sum_{p \in P} s(\| \vec{p} - \vec{x} \|) (\vec{p} - \vec{x})
,\]
where \( s : \R^+ \to \R^+ \) is some function that, intuitively, should be
small near \( 0 \), increase until a value somewhat close to \( 0 \), and then
taper off. We desire this behavior as we don't need to move colors that are too
close or too far away from the colorscheme point, but we want to move the ones
that are somewhat close. A good candidate family of functions are the polynomial exponentional functions, where we choose
\[
    s(r) = C r^{n} \exp(-ar)
.\]
Indeed, we pick \( C = 1, n = 1, a = 3.5 \), which yields a rather pleasant
result after darkening the resulting image after filtering.

\end{document}
